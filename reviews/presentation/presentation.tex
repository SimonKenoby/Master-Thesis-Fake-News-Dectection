\documentclass{beamer}
\usepackage{graphics}
\usepackage[autoplay,autoresume,loop]{animate}
\usepackage{subcaption}
\usepackage{pgfpages}

\setbeameroption{show notes on second screen=right} % Both
\setbeamertemplate{note page}{\pagecolor{yellow!5}\insertnote}\usepackage{palatino}

\usetheme{metropolis}           % Use metropolis theme
\title{Fake News Detection Using Machine Learning}
\date{September 10, 2019}
\author{Author: Simon Lorent \\ Supervisor: Aswhin Ittoo}
\institute{University Of Liège}
\begin{document}
  \maketitle
  \section{Introduction}

  \begin{frame}{Fake news used to influence elections}
  2016 US presidential elections (Allcott et al.)\cite{Allcott2017}
	  \begin{itemize}
	 \item $62\%$ of US citizens get their news for social medias\cite{gottfried2016news}
	 \item Fake news had more share on Facebook than mainstream news\cite{silverman2016teens}.
	\end{itemize}
	\note[item]{Expliquer l'inpact des fake news sur les elections}
	\note[item]{La suite définit ce que sont les fake news}
  \end{frame}

  \begin{frame}{Definition}
  \newtheorem{def:fake_news}{Definition}
	\begin{def:fake_news}
	Fake news is a news article that is intentionally and verifiable false\cite{shu2017fake}
	\end{def:fake_news}
	\note[item]{Dire que l'on peut caracteriser les fake news de plusieurs façon: le contenu et le context}
  \end{frame}

  \begin{frame}[allowframebreaks]{Fake News Characterisation}
News content features:
  	\begin{itemize}
 \item \textbf{Source}: Where does the news come from, who wrote it, is this source reliable or not.
 \item \textbf{Headline}: Short summary of the news content that try to attract the reader.
 \item \textbf{Body Text}: The actual text content of the news.
 \item \textbf{Image/Video}: Usualy, textual information is agremented with visual information such as images, videos or audio.  
\end{itemize}
	\newpage
  Social context features:
  \begin{itemize}
 \item \textbf{Expert-oriented}: relies on experts, such as journalists or scientists, to assess the news content.
 \item \textbf{Crowdsourcing-oriented}: relies on the wisdom of crowd that says that if a sufficiently large number of persons say that something is false or true then it should be.
 \item \textbf{Computational-oriented}: relies on automatic fact checking, that could be based on external resources such as DBpedia.
\end{itemize}
  \end{frame}
  \begin{frame}[allowframebreaks]{Bibliography}
    \bibliographystyle{unsrt}
	\bibliography{../references/references.bib}
  \end{frame}
\end{document}