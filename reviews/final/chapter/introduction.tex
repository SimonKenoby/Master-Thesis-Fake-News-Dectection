
\chapter{Introduction}
\section{What are fake news?}
\paragraph{} Fake news have quickly become a society problem, being used to propagate false or rumorus informations in order to change behaviors of peoples. Before stating to work on detecting fake news, it is needed to first understand what they are. It have been show that propagation of fake news have had a non negligable influence of 2016 US presidential elections\cite{Allcott2017}. A few facts on fake news in the United States: 

\begin{itemize}
	\item $62\%$ of US citizen get there news for social medias\cite{gottfried2016news}
	\item Fake news had more share on facebook than mainstream news\cite{silverman2016teens}.
\end{itemize}

Fake news have also been used in order to influence the referendum in the United Kingdom for the "Brexit".

%TODO: Define fake news
%TODO: Expliquer les differentes méthodes de détection de fake news dans le paper Fake News Detection on Social Media: A Data Mining Perspective 1708.01967v3.pdf
\paragraph{}Shu et al\cite{shu2017fake} provides multiples features that can be used for fake news detection: 

\begin{itemize}
\item News Content Features:
	\begin{itemize}
		\item Linguistic-based
		\item Visual-based
	\end{itemize}
	\item Social Context Features:
	\begin{itemize}
		\item User-based
		\item Post-based
		\item Network-based
	\end{itemize}
\end{itemize} 

They also provides multiples approaches in order to build the model. 

\begin{itemize}
	\item News Content Models:
	\begin{itemize}
		\item Knowledge-based
		\item Style-based
	\end{itemize}
	\item Social Context Models:
	\begin{itemize}
		\item Stance-based
		\item Propagation-based
	\end{itemize}
\end{itemize}

\section{Datasets}
\subsection{Fake News Corpus}
\paragraph{}
This works uses multiples corpus in order to train and test different models. The main corpus used for training is called Fake News Corpus\cite{Szpakowski}. This corpus have been automatically crawled using \url{opensources.co} labels. In other words, domains have been labeled with one or more labels in \begin{itemize}
	\item Fake News
	\item Satire
	\item Extreme Bias
	\item Conspiracy Theory
	\item Junk Science
	\item Hate News
	\item Clickbait
	\item Proceed With Caution
	\item Political
	\item Credible
\end{itemize}
\paragraph{}
These annotations have been provided by crowdsourcing, which means that they might not be exactly accurate, but are expected to be close to the reality. Because this works focus on fake news detection against reliable news, only the news labels as fake and credible have been used. 

\textbf{TODO: Expliquer comment le dataset a été nettoyé et mis dans une base de données afin d'augmenter les performances. }

\subsection{Fake News Net}
\paragraph{}
The second dataset used is fake news net\cite{shu2018fakenewsnet,shu2017exploiting,shu2017fake}. This corpus is made of news from two different sources, PolitiFact and GossipCop. An older version also provide news from BuzzFeed. News are categorized in two classes: fake and non fake. Being quite smaller than fake news corpus, this dataset will be used as a test dataset. 

\subsection{Liar, Liar Pants on Fire}
\paragraph{}
The third and last dataset is \textbf{Liar, Liar Pants on Fire} dataset\cite{Wang2017}, which is a collection of twelve thousand small sentences collected from various sources and hand labeled. They are devided in six classes:
\begin{itemize}
	\item pants-fire
	\item false
	\item barely-true
	\item half-true
	\item mostly-true
	\item true
\end{itemize} 
\paragraph{}
This set will be used a second test set. Because in this case there are six classes againt two in the other cases, a threshould should be used in order to fix which one will be concidered as true or false. 
\paragraph{}
It should be noted that this one differ from the two other dataset is it is composed only on short sentences, and thus it should not be expected to have very good results on this dataset for models trained on Fake News Corpus which is made of full texts. 

\section{State of the Art}