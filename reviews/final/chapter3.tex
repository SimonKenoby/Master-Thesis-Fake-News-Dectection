\chapter{TODO: find a title}
\section{Introduction}
\paragraph{} In this chapter, we will focus on the more traditional methods used in natural language processing such as Naïve-Bayes, decision trees, linear SVM and others. These will serve as baseline for comparing the performances of the more two advanced models that will be analyse later on: LSTM and Attention Mechanism. The first thing to do when working with text is the do words and texts embedding, indeed, in order to use machine learning algorithms on texts, a mathematical representation of these texts is required. 

\section{Words and text embedding}
\paragraph{}  As explained before, text need to be represented in a way that gives more meaningful information than a simple sequence of bits, which have additional drawbacks such that for a given word, the sequence of bits representing it depends on the coding.
\paragraph{} The first and simplest coding that come to mind is a one-hot encoding: a matrix $M$ of size number of texts $\times$ number of words where $M_{ij} = 1$ if the word $j$ is present in the text  $i$ and $0$ in the other case. But this is till not enougth as each word is given the same weight, no matter how often it appears in the text. 

\paragraph{} In order to overcome this problem, term-frequency might be used, that is, rather than setting $M_{ij}$ to 0 or 1 we set it to the number of time it appears in the text.  